%%%%%%%%%%%%%%%%%%%%%%%%%%%%%%%%%%%%%%%%%%%%%%%%%%%%%%%%%%%%%%%%%%%%%%%%%%%%%%%%

\pagestyle{empty}
\begin{abstract}
  Este trabajo presenta el desarrollo de un sistema de captura y procesado de datos IMU utilizando un dispositivo Android y un entorno de análisis en Python. La solución implementada permite adquirir muestras por red local, almacenar sesiones temporales y aplicar un pipeline de procesado orientado a estimar movimiento vertical relativo a partir de aceleración.

  El procesado incluye estimación de frecuencia de muestreo, separación de componente gravitatoria mediante filtrado pasa-bajos, filtrado en banda para aislar la dinámica de interés y cálculo de altura relativa por integración espectral. Para reducir ruido y falsos positivos se añadieron mecanismos de umbral mínimo de cambio y confirmación temporal por persistencia de muestras consecutivas.

  Los resultados obtenidos muestran que la arquitectura propuesta es viable y reproducible, y que la selección de parámetros de filtrado condiciona de forma decisiva la sensibilidad del sistema. Como trabajo futuro se plantea incorporar corrección por orientación del dispositivo y automatizar la calibración de umbrales para distintos escenarios de medida.

  \vspace*{25pt}
  \begin{segundoresumo}
    Este traballo presenta o desenvolvemento dun sistema de captura e procesado de datos IMU empregando un dispositivo Android e unha contorna de análise en Python. A solución implementada permite adquirir mostras por rede local, almacenar sesións temporais e aplicar unha cadea de procesado orientada a estimar movemento vertical relativo a partir da aceleración.

    O procesado inclúe estimación da frecuencia de mostraxe, separación da compoñente gravitatoria mediante filtrado pasa-baixos, filtrado en banda para illar a dinámica de interese e cálculo da altura relativa por integración espectral. Para reducir ruído e falsos positivos engadíronse mecanismos de limiar mínimo de cambio e confirmación temporal por persistencia de mostras consecutivas.

    Os resultados obtidos mostran que a arquitectura proposta é viable e reproducible, e que a selección de parámetros de filtrado condiciona de maneira decisiva a sensibilidade do sistema. Como traballo futuro proponse incorporar corrección por orientación do dispositivo e automatizar a calibración de limiares para distintos escenarios de medida.
  \end{segundoresumo}
\vspace*{25pt}
\begin{multicols}{2}
\begin{description}
\item [\palabraschaveprincipal:] \mbox{} \\[-20pt]
  \begin{itemize}
    \item Unidad de medida inercial (IMU)
    \item Android
    \item Procesado de señal
    \item Acelerómetro
    \item Estimación de altura relativa
    \item Filtrado en frecuencia
    \item Análisis temporal
  \end{itemize}
\end{description}
\begin{description}
\item [\palabraschavesecundaria:] \mbox{} \\[-20pt]
  \begin{itemize}
    \item Unidade de medida inercial (IMU)
    \item Android
    \item Procesado de sinal
    \item Acelerómetro
    \item Estimación de altura relativa
    \item Filtrado en frecuencia
    \item Análise temporal
  \end{itemize}
\end{description}
\end{multicols}

\end{abstract}
\pagestyle{fancy}

%%%%%%%%%%%%%%%%%%%%%%%%%%%%%%%%%%%%%%%%%%%%%%%%%%%%%%%%%%%%%%%%%%%%%%%%%%%%%%%%

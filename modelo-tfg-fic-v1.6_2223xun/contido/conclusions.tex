\chapter{Conclusiones}
\label{chap:conclusions}

\lettrine{E}{l} trabajo realizado permite concluir que es viable construir un sistema ligero de captura y procesado IMU con herramientas accesibles, manteniendo una cadena completa desde adquisición en Android hasta análisis técnico en escritorio.
Esta fase constituye la validación metodológica inicial de una solución mayor orientada a boya instrumentada.

\section{Conclusiones principales}

\begin{itemize}
  \item Se implementó una solución funcional extremo a extremo para captura y análisis de aceleración vertical relativa.
  \item Se consiguió una infraestructura reproducible de pruebas basada en sesiones temporales y scripts parametrizables.
  \item El análisis por etapas facilitó identificar el impacto de cada filtro y corregir configuraciones demasiado agresivas.
  \item La incorporación de confirmación temporal de cambios redujo falsos positivos asociados a ruido de muestra aislada.
  \item Se establecieron criterios de comparación para la transición a una plataforma embebida ESP32+LoRa.
\end{itemize}

\section{Aportaciones}

Las aportaciones más relevantes del trabajo son:

\begin{itemize}
  \item Diseño modular de scripts de captura y procesado.
  \item Soporte de formatos de entrada/salida en distintos entornos de ejecución.
  \item Estrategia práctica de validación basada en umbral mínimo y persistencia temporal.
  \item Documentación técnica incremental útil para evolución futura del proyecto.
\end{itemize}

\section{Trabajo futuro}

Como evolución natural del TFG se proponen las siguientes líneas:

\begin{itemize}
  \item Implementación completa de la Fase 2 en ESP32 con IMU y transmisión LoRa.
  \item Comparativa experimental Android vs ESP32+LoRa con métricas homogéneas.
  \item Proyección de aceleración al eje vertical global usando información de orientación del dispositivo.
  \item Calibración automática de umbrales a partir de un tramo inicial en reposo.
  \item Generación automática de informes por sesión con métricas y figuras.
  \item Validación con campañas experimentales controladas y comparación frente a referencia externa.
\end{itemize}

\chapter{Introducción}
\label{chap:introducion}

\lettrine{E}{ste} trabajo fin de grado aborda el diseño e implementación de una boya instrumentada capaz de caracterizar el oleaje mediante medidas inerciales.
El desarrollo se planteó en dos fases:
una fase inicial con Android como prototipo rápido de validación
y una fase posterior con plataforma embebida ESP32 + IMU + LoRa.
El objetivo principal es estimar movimiento vertical relativo a partir de aceleración medida en el eje \texttt{z}, aplicando técnicas de filtrado y análisis en frecuencia para reducir ruido y obtener métricas interpretables.

\section{Contexto y motivación}

En entornos reales, las señales de aceleración incluyen ruido, variaciones de orientación, pequeñas derivas y componentes gravitatorias que dificultan la interpretación directa de la medida~\cite{TittertonWeston2004}. Durante el desarrollo se observó que una aproximación naive basada en leer directamente \texttt{az} produce falsas variaciones de altura cuando el dispositivo está en reposo o cuando cambia ligeramente la inclinación. Esta limitación motiva la construcción de un pipeline de procesado robusto y parametrizable.

En la fase actual, el sistema desarrollado permite:

\begin{itemize}
  \item Capturar datos IMU desde Android por HTTP en red local.
  \item Guardar sesiones temporales de medida para análisis offline.
  \item Procesar las señales con distintos filtros y estrategias de validación temporal.
  \item Visualizar la evolución de cada etapa para entender el impacto de cada parámetro.
\end{itemize}

Como siguiente fase, se plantea migrar la captura a ESP32 para aumentar la frecuencia de muestreo efectiva hasta valores cercanos a \(1000\) Hz en adquisición local, y transportar telemetría por LoRa.

\section{Objetivos}

\subsection{Objetivo general}

Implementar y validar un sistema extremo a extremo para estimación de movimiento vertical relativo a partir de datos IMU, priorizando reproducibilidad experimental y capacidad de ajuste de parámetros.

\subsection{Objetivos específicos}

\begin{itemize}
  \item Diseñar una arquitectura de comunicación Android--servidor basada en HTTP para validación temprana.
  \item Implementar un módulo de captura temporal con persistencia en \texttt{CSV}/\texttt{NPZ}.
  \item Construir un pipeline de procesado con separación de gravedad, filtrado en banda e integración espectral.
  \item Incorporar mecanismos de reducción de ruido: umbral mínimo y confirmación por persistencia temporal.
  \item Evaluar el comportamiento del sistema con sesiones de prueba y documentar parámetros efectivos.
  \item Definir y ejecutar una comparación entre plataforma Android y plataforma ESP32+LoRa en términos de calidad de medida y viabilidad de despliegue en boya.
\end{itemize}

\section{Alcance y limitaciones}

El alcance del trabajo se centra en estimación de movimiento vertical \textit{relativo} sobre sesiones cortas de captura. No se plantea una estimación absoluta de posición global ni navegación inercial completa, dado que ello requeriría fusión avanzada con magnetómetro/GNSS y modelos de error más complejos~\cite{Farrell2008}.

Las limitaciones principales detectadas son:

\begin{itemize}
  \item Sensibilidad a orientación del dispositivo.
  \item Dependencia fuerte de la banda de frecuencias elegida.
  \item Atenuación de movimientos lentos cuando los filtros son demasiado restrictivos.
  \item Frecuencia de muestreo limitada en Android (aprox. \(50\) Hz en el prototipo actual).
\end{itemize}

\section{Metodología de desarrollo}

El trabajo se realizó de forma incremental:

\begin{enumerate}
  \item Fase 1: prototipo Android para captura por red local y validación del pipeline.
  \item Captura estructurada de sesiones de 30 segundos.
  \item Procesado offline con métricas de validación y visualización.
  \item Ajuste de parámetros con experimentación controlada.
  \item Fase 2: migración prevista a ESP32+IMU+LoRa y comparación de plataformas.
  \item Documentación de resultados y consolidación de pipeline.
\end{enumerate}

\section{Estructura de la memoria}

La memoria se organiza de la siguiente forma:

\begin{itemize}
  \item En el Capítulo~\ref{chap:demo} se describe el diseño técnico, la implementación del sistema, el pipeline de procesado y los resultados de pruebas.
  \item En el Capítulo~\ref{chap:conclusions} se presentan conclusiones, aportaciones y líneas futuras.
  \item En el Apéndice~\ref{chap:adicional} se incluye material de soporte y reproducibilidad experimental.
\end{itemize}

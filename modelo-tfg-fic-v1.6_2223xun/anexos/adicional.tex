\chapter{Material adicional}
\label{chap:adicional}

\lettrine{E}{n} este apéndice se recoge material de apoyo para reproducir el entorno experimental y el flujo de trabajo técnico del proyecto.

\section{Scripts principales}

\begin{itemize}
  \item \texttt{capture\_http\_imu.py}: recepción HTTP y guardado de sesiones.
  \item \texttt{process\_imu\_session.py}: procesado offline con métricas y visualización.
  \item \texttt{process\_imu\_step\_by\_step.py}: visualización por etapas del pipeline.
\end{itemize}

\section{Comandos de referencia}

\subsection{Captura}

\begin{lstlisting}[language=bash]
python3 capture_http_imu.py --host 0.0.0.0 --port 8001 --seconds 30
\end{lstlisting}

\subsection{Procesado rápido sin gráfica}

\begin{lstlisting}[language=bash]
python3 process_imu_session.py session_YYYYMMDD_HHMMSS_30s.csv --no-plot
\end{lstlisting}

\subsection{Procesado paso a paso con validación temporal}

\begin{lstlisting}[language=bash]
python3 process_imu_step_by_step.py session_YYYYMMDD_HHMMSS_30s.csv \
  --mode step4 --g-fc 0.03 --fmin 0.02 --fmax 1.20 \
  --height-min-change 0.00005 --height-step-min 0.000001 --confirm-samples 6
\end{lstlisting}

\section{Datos de prueba}

Se trabajó con múltiples sesiones de 30 segundos en formato CSV, incluyendo capturas de reposo y capturas con movimiento vertical manual. La bitácora de resultados y ajustes se mantiene en \texttt{INFORME\_IMU.md}.
